\chapter{Introduction}
With the increasing amount of data nowadays, it has become extremely important to develop tools that allow us to treat it better. In the case of images, unfortunately a lot of them are still taken with old technology leading to a weak resolution. Recently, a new theory called super-resolution has emerged. Its aim is to reconstruct an image of better quality from a poor quality image.\par 

The paper we will study here is called "Super-Resolution Off the Grid" and has been published by Qingqing Huang and Sham M. Kakade in September 2015. From a theoretical point of view, super-resolution is the problem of recovering a superposition of point sources using bandlimited measurements, which may be corrupted with noise.\par 

The goal of this project is to get familiar with the theory of super-resolution along with all important mathematical preliminaries and study the main result presented in the above mentioned paper. The idea is then to compare this work with other related ones and discuss the efficiency of the main algorithm. Ultimately, it would be a nice addition to start working on an improvement of the main algorithm in order to reduce its complexity and thus make it run faster.