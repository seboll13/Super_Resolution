\chapter{Introduction}
With the increasing amount of data nowadays, it has become extremely important to develop tools that allow us to treat it better. In the case of images, today's technology often neglects important information located in the image. For example, a microscope's hardware will usually act as a low pass filter so that all the high frequency points of some image will be ignored. In essence, the resulting image will contain some noise and thus sometimes miss fine critical details. Recently however, new algorithms that are capable of recovering the missing details of some fine image given only access to some coarse measurements of a signal, have been designed. This process is called super-resolution.\par 

The paper we will study here is called "Super-Resolution Off the Grid" and has been published by Qingqing Huang and Sham M. Kakade in September 2015. From a theoretical point of view, super-resolution is the problem of recovering a superposition of point sources using bandlimited measurements, that often may be corrupted by noise. Essentially, suppose we are given a collection of point sources that lie on the $d$-dimensional plane, to which we have no access whatsoever. Those point sources can often be seen as brightness spikes in a given image. We use measurements from the Fourier domain that we wish to super-resolve to recover good enough estimates of the point sources.\par 

The goal of this project is to get familiar with the theory of super-resolution along with all important mathematical preliminaries and study the main result presented in the aforementioned paper. The idea is then to compare this work with other related ones and discuss the efficiency of the main algorithm. Ultimately, it would be a nice addition to start working on an improvement of the main algorithm in order to reduce its sample complexity from quadratic space to the information theoretical linear one.