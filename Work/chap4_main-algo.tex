\chapter{Main algorithm}
Let us recall what the goal of the problem is. Broadly speaking, super-resolution aims to recover a superposition of point sources using bandlimited measurements that may be corrupted with noise. In this setting, the created algorithm works in the Fourier domain where each of the $k$ points of the $d$-dimensional plane are separated by a distance at least $\Delta$. Hence, the frequencies of the Fourier measurements are bounded by $O(1/\Delta)$. The idea of the procedure is to take random bandlimited measurements with cutoff frequency bounded by $\Omega(\sqrt{d}/\Delta)$ and perform Tensor decomposition to recover the estimates of the point sources.\paragraph{Input and output} The algorithm takes as input a cutoff frequency $R$, the defined number of measurements $m$ and a noisy measurement function $\tilde{f}(\cdot)$ and outputs the set of estimates $$\{\widehat{w}_j,\widehat{\mu}^{(j)}:j\in[k]\},$$ where the $\widehat{w}_j$'s are the complex weight coefficients and the $\widehat{\mu}^{(j)}$'s are the estimates of the point sources. Note that in case the noise is non-existant (i.e. when $\epsilon_z=0$), the parameters are recovered exactly. Otherwise, stable recovery implies that the estimates are such that $$\min_\pi\max\left\{||\widehat{\mu}^{(j)}-\mu^{(\pi(j))}||_2:j\in[k]\right\}\leq poly(d,k)\epsilon_z.$$ In words, the estimates of the point sources differ from their real values by at most the noise $\epsilon_z$ scaled by a polynomial function that depends on the number of dimensions $d$ and the number of point sources $k$.
\paragraph{Measurements}