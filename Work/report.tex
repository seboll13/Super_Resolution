\documentclass[11pt,titlepage]{report}
\usepackage{Preamble}

\begin{document}

\begin{titlepage}
    \newgeometry{margin=3cm}
	\centering
    \includegraphics[width=0.5\linewidth]{images/EPFL.png}\\[0.25cm] 	% University Logo
    \textsc{\LARGE École Polytechnique Fédérale de Lausanne}\\ \vspace{\fill}
    \textbf{\textsc{\fontsize{30}{30}\selectfont Super-Resolution Off the Grid}}\\ \vspace{\fill}		
	\textsc{\LARGE EPFL - Semester Project in Computer Science}\\[0.4cm]
	\rule{\linewidth}{0.2 mm} \\[0.5 cm]
	Sébastien Ollquist \\[2cm] \today
\end{titlepage}
\restoregeometry

\thispagestyle{numberonly}
\begin{summary}
\section*{Abstract}
Super-Resolution is the tool that allows us to upgrade the quality of images. The goal of this project is to study a recent paper on super-resolution and discuss the main algorithm in order to eventually find an improvement using information theoretical bounds.
\end{summary}

% \begin{invsummary}
% Why not add this here ?
% \end{invsummary}

\begin{fquote}[Albert Einstein]
    In theory, theory and practice are the same. In practice, they are not.
\end{fquote}


\chapter{Introduction}

\chapter{Mathematical refresher}

\chapter{Results and discussion}

\chapter{Improvement}

\chapter{Conclusion}

\clearpage
\pagestyle{numberonly}
\printbibliography

\end{document}