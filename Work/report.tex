\documentclass[11pt,titlepage]{report}
\usepackage{Preamble}

\newtheorem{theorem}{Theorem}
\newtheorem{corollary}[theorem]{Corollary}
\newtheorem{lemma}[theorem]{Lemma}
\newtheorem{claim}[theorem]{Claim}
\newtheorem{definition}[theorem]{Definition}
\newtheorem{fact}[theorem]{Fact}
\newenvironment{proof}{{\bf Proof\quad}}{\hfill\rule{2mm}{2mm}}
\DeclareMathOperator*{\cond}{cond_2}

\begin{document}

\begin{titlepage}
    \newgeometry{margin=3cm}
	\centering
    \includegraphics[width=0.5\linewidth]{images/EPFL.png}\\[0.25cm] 	% University Logo
    \textsc{\LARGE École Polytechnique Fédérale de Lausanne}\\ \vspace{\fill}
    \textbf{\textsc{\fontsize{30}{30}\selectfont Super-Resolution Off the Grid}}\\ \vspace{\fill}		
	\textsc{\LARGE Semester Project in Computer Science}\\[0.4cm]
	\rule{\linewidth}{0.2 mm} \\[0.5 cm]
	Sébastien Ollquist \\[2cm] \today
\end{titlepage}
\restoregeometry

\thispagestyle{numberonly}
\begin{abstract}
    Super-resolution is the tool that fundamentally allows us to increase the resolution of low resolution images. The original problem has both a practical and a theoretical meaning, the latter in which we wish to recover a superposition of point sources on a $d$-dimensional plane having merely access to bandlimited Fourier measurements. The goal of this project is to study a recent paper on theoretical super-resolution and discuss the main procedure that aims at recovering a collection of source points on a plane. Ultimately, some intuition will be presented on how the procedure could be improved such that its sample complexity is reduced.
\end{abstract}
\chapter*{Acknowledgements}
I would like to take the opportunity to thank the people that contributed in some way to the progression of this project. First of all, my thanks go to my project supervisor Prof. Michael Kapralov who suggested me this very interesting paper and who supervised all my work during this semester. He managed to push me in some way to give the very best of me and even though theory is hard, I would without hesitation do this again. I would also like to express my gratitude to my co-supervisor Navid Nouri who supported me from the very begining of the project and who gave me some very helpful advice all along. Finally, I would like to thank Kshiteej Sheth and Mikhail Makarov for some very interesting discussions towards the end of the project.
\addcontentsline{toc}{chapter}{Acknowledgements} 
\begin{fquote}[Albert Einstein]
    In theory, theory and practice are the same. In practice, they are not.
\end{fquote} 
\newpage

\tableofcontents

\chapter{Introduction}
With the increasing amount of data nowadays, it has become extremely important to develop tools that allow us to treat it better. In the case of images, today's technology often neglects important information located in the image. For example, a microscope's hardware will usually act as a low pass filter so that all the high frequency points of some image will be ignored. In essence, the resulting image will contain some noise and thus sometimes miss fine critical details. Recently however, new algorithms that are capable of recovering the missing details of some fine image given only access to some coarse measurements of a signal, have been designed. This process is called super-resolution.\par 

The paper we will study here is called "Super-Resolution Off the Grid" and has been published by Qingqing Huang and Sham M. Kakade in September 2015. From a theoretical point of view, super-resolution is the problem of recovering a superposition of point sources using bandlimited measurements, that often may be corrupted by noise. Essentially, suppose we are given a collection of point sources that lie on the $d$-dimensional plane, to which we have no access whatsoever. Those point sources can often be seen as brightness spikes in a given image. We use measurements from the Fourier domain that we wish to super-resolve to recover good enough estimates of the point sources.\par 

The goal of this project is to get familiar with the theory of super-resolution along with all important mathematical preliminaries and study the main result presented in the aforementioned paper. The idea is then to compare this work with other related ones and discuss the efficiency of the main algorithm. Ultimately, it would be a nice addition to start working on an improvement of the main algorithm in order to reduce its sample complexity from quadratic space to the information theoretical linear one.
\chapter{Mathematical refresher}
\section{A mathematical theory of super-resolution}
We consider $k$ point sources in $d$ dimensions, where the points are separated by a distance at least $\Delta$ (in Euclidean distance). The $d$-dimensional signal $x(t)$ can be modeled as a weighted sum of $k$ Dirac measures in $\mathbb{R}^d$ as $$x(t)=\sum_{j=1}^k w_j\delta_{\mu^{(j)}},$$ where the $\mu^{(j)}$'s are the point sources in $\mathbb{R}^d$ and $w_j\in\mathbb{C}$ the weights such that $|w_j|<C$ for every $j\in[k]$ and some absolute constant $C>0$.

\section{Main tools}
The main mathematical tools that are needed to understand the paper essentially lie around the subject of linear algebra. They include operations related to vectors, matrices and tensors. We also require some probabilistic analysis tools since the algorithm is partly random. In this chapter, we introduce those tools, prove the more important results and closely relate them to the paper. 
\subsection{Generalised eigenvalue problem}
Before introducing the generalised eigenvalue problem, it is important to recall what a condition number is for a particular matrix. Suppose for instance we have a matrix $X\in\mathbb{R}^{m\times n}$. We let $\lambda_1,\ldots,\lambda_n$ be the eigenvalues of $X^TX$ (with repetitions) and arrange them so that $\lambda_1\geq\ldots\geq\lambda_n\geq 0$. Then, the $\sigma_1\geq\ldots\geq\sigma_n\geq 0$ such that $\sigma_i=\sqrt{\lambda_i}$ are called the \textit{singular values} of $X$. Define $\sigma_{max}(X)=\sigma_1$ and $\sigma_{min}(X)=\sigma_n$. We then define the condition number of a matrix to be the ratio between the largest and the smallest singular value of $X$. That is 
\begin{equation}
    \cond(X)=\sigma_1/\sigma_n=\frac{\sigma_{max}(X)}{\sigma_{min}(X)}.
\end{equation}
The above factor governs the noise tolerance of the generalised eigenvalue problem, i.e. it is the measure of sensitiveness of a matrix to arbitrary perturbations. Taking its limiting value will allow us to state whether or not the main algorithm achieves stable recovery of the point sources.\par
The goal of an eigenvalue problem is to simply find the eigenvalues of a particular matrix $A$. To do so, we generally solve the following equation: $AU=UV\Rightarrow AUU^T=UVU^T\Rightarrow A=UVU^T=UVU^{-1}$, where $\mathbb{R}^{d\times d}\ni U:=(u_1,\ldots,u_d)$ are the eigenvectors and $\mathbb{R}^{d\times d}\ni V:=Diag\left[(\lambda_1,\ldots,\lambda_d)^T\right]$ the eigenvalues. However, in the generalised version of this problem, we add another random matrix $B$ such that the problem becomes $$AU=BUV\Rightarrow AUU^T=BUVU^T\Rightarrow A=BUVU^T=BUVU^{-1},$$ where $U$ again are the eigenvectors and $V$ the eigenvalues. Note that $A$ and $B$ are both symmetric. We form the pair $(A,B)$ called the \textit{pencil} and the pair $(U,V)$ called the \textit{eigenpair}. Therefore, we introduce a first version of what is called the matrix pencil method, in which given a pair of matrices $(A,B)$, we wish to find the generalised eigenvalues $\lambda$ for which there is a vector ${\bf x}$ such that $A{\bf x}=\lambda B{\bf x}$. Note that the eigenvalues of $A$ are the solution to the generalised eigenvalue problem where $B=I$.

\section{Tensor decomposition}
\subsection{Gentle introduction}
A tensor is a generalisation of a matrix to more than two dimensions. We can think of a tensor as a point in $\mathbb{C}^{m_1\times\ldots\times m_k}$ where $k$ is the order of the tensor. Most of the time here, $k=3$ since three dimensions suffice for our analysis. Note that if $T$ is an order three tensor of dimensions $m_A\times m_B\times m_C$, we can view it as a collection of $m_C$ matrices of size $m_A\times m_B$ stacked on top of each other.\par
We define the \textit{rank} of a tensor $V$ as the minimum $r$ such that we can write $V$ as the sum of rank one tensors. A rank one tensor will be decomposed in the form of a tensor product of three matrices $A$, $B$ and $C$ as $V=A\otimes B\otimes C$. The above product is element-wise defined as $V_{i_1,i_2,i_3}=\sum_{j=1}^k A_{i_1,j}B_{i_2,j}C_{i_3,j}$.\par 
An alternative definition is given using the notion of a multi-linear mapping. Namely, for given dimensions $m_A$, $m_B$, $m_C$, the mapping $V(\cdot,\cdot,\cdot):\mathbb{C}^{m\times m_A}\times\mathbb{C}^{m\times m_B}\times\mathbb{C}^{m\times m_C}\to\mathbb{C}^{m_A\times m_B\times m_C}$ is defined as: $$\left[V(X_A,X_B,X_C)\right]_{i_1,i_2,i_3}=\sum_{j_1,j_2,j_3\in[m]}V_{j_1,j_2,j_3}[X_A]_{j_1,i_1}[X_B]_{j_2,i_2}[X_C]_{j_3,i_3}.$$
We can verify that for a particular vector $a\in\mathbb{C}^m$, the projection $V(I,I,a)$ of $V$ along the 3rd dimension is $V(I,I,a)=ADiag(C^T a)B^T$ as long as $V$ admits a tensor decomposition $V=A\otimes B\otimes C$. Indeed,
\begin{flalign*}
    \left[V(I,I,a)\right]_{i_1,i_2,i_3}&=\sum_{j_1,j_2,j_3\in[m]}V_{j_1,j_2,j_3}[I]_{j_1,i_1}[I]_{j_2,i_2}[a]_{j_3,i_3}\\
    &=\sum_{j_1,j_2,j_3\in[m]}\sum_{n\in[k]} A_{j_1,n}B_{j_2,n}C_{j_3,n}[I]_{j_1,i_1}[I]_{j_2,i_2}[a]_{j_3,i_3}\\
    &=...\\
    &=Diag(C^T a)(A\otimes B)\\
    &=ADiag(C^T a)B^T
\end{flalign*}
\subsection{Jennrich's algorithm}
\chapter{Initial work}
\section{Prony's method}

\section{Univariate case: adaptation of Prony's method}
The initial setting is as follows. We construct two $m\times m$ complex valued Hankel matrices $H_0$ and $H_1$, that is, matrices such that their skew-diagonals\footnote{A skew-diagonal is the diagonal in the North-East direction.} are constants. We have $D_w\in\mathbb{C}^{k\times k}$ where $[D_w]_{j,j}=w_j$ and $D_\mu\in\mathbb{C}^{k\times k}$ where $[D_\mu]_{j,j}=e^{i\pi\mu^{(j)}}$. We furthermore have a Vandermonde matrix $V_m\in\mathbb{C}^{m\times k}$ defined as $$V_m=
\begin{pmatrix}
    1&\ldots&1\\
    \big(e^{i\pi\mu^{(1)}}\big)^1&\ldots&\big(e^{i\pi\mu^{(k)}}\big)^1\\
    \vdots&\ldots&\vdots\\
    \big(e^{i\pi\mu^{(1)}}\big)^{m-1}&\ldots&\big(e^{i\pi\mu^{(k)}}\big)^{m-1}
\end{pmatrix}$$ We construct a 3rd order tensor $F\in\mathbb{C}^{m\times m\times 2}$ in which $F_{i,i',j}=[H_{j-1}]_{i,i'}$ for $j=1,2$ and $i,i'\in[m]$.
\begin{fact}
    In the setting above, $F$ admits the unique rank $k$ tensor decomposition $F=V_m\otimes V_m\otimes (V_2D_w)$.
\end{fact}
\begin{proof}
    We start by computing the product $V_2D_w$. By definition of $V_m$, we have that $[V_2]_{r,c}=\big(e^{i\pi\mu^{(c)}}\big)^{r-1}$ for $r\in[2]$ and $c\in[k]$. Multiplying with the diagonal matrix $D_w$ yields $$[V_2D_w]_{r,c}=w_c\big(e^{i\pi\mu^{(c)}}\big)^{r-1},\quad\forall r\in[2],c\in[k].$$
    We aim to show that 
    \begin{flalign*}
        F_{i_1,i_2,i_3}&=\sum_{n=1}^k[V_m]_{i_1,n}[V_m]_{i_2,n}[V_2D_w]_{i_3,n}\\
        &=\sum_{n=1}^k\big(e^{i\pi\mu^{(n)}}\big)^{i_1-1}\big(e^{i\pi\mu^{(n)}}\big)^{i_2-1}w_n\big(e^{i\pi\mu^{(n)}}\big)^{i_3-1}\\
        &=\sum_{n=1}^k w_n\big(e^{i\pi\mu^{(n)}}\big)^{i_1+i_2+i_3-3},
    \end{flalign*} for $i_1,i_2\in[m]$ and $i_3\in[2]$. It can be verified that the measurements that form both Hankel matrices are given by $$f(s)=\sum_{j\in[k]}w_j\big(e^{i\pi\mu^{(j)}}\big)^s,$$ with $s=0,\ldots,2m-1$ for $H_0$ and $s=1,\ldots,2m$ for $H_1$. For the details of the proof, see Appendix A. Then, since $F_{i_1,i_2,i_3}=[H_{i_3-1}]_{i_1,i_2}$ by definition of $F$ for $i_1,i_2\in[m]$ and $i_3\in[2]$, we directly get that 
    $$F_{i_1,i_2,i_3}=\sum_{n=1}^k w_n\big(e^{i\pi\mu^{(n)}}\big)^{i_1+i_2+i_3-3},$$ as required.
\end{proof}
\section{Multivariate toy case}
We follow the definitions given in the paper. We have $D_w=Diag(w_j)\in\mathbb{C}^{k\times k}$ and $[V_d]_{r,c}=e^{i\pi\mu_r^{(c)}}\in\mathbb{C}^{d\times k}$ the latter in which $\mu_n^{(i)}$ are i.i.d. $\sim Unif([-1,+1])$. Furthermore, we have $F_{n_1,n_2,n_3}=f(s)\big|_{s=e_{n_1}+e_{n_2}+e_{n_3}},$ for all $n_1,n_2,n_3\in[d]$. We have the following fact:
\begin{fact}
    In the setting above, $F$ admits the tensor decomposition $F=V_d\otimes V_d\otimes(V_dD_w)$.
\end{fact}
\begin{proof}
    We wish to show that $f(e_1+e_2+e_3)=\sum_{j=1}^k w_je^{i\pi(\mu_1^{(j)}+\mu_2^{(j)}+\mu_3^{(j)})}$. To do so, we first compute the matrix product $V_dD_w$. Since $[V_d]_{r,c}=e^{i\pi\mu_r^{(c)}}$ for $r\in[d],c\in[k]$ and $D_w=Diag(w_j)$, we directly have that $$[V_dD_w]_{r,c}=w_ce^{i\pi\mu_r^{(c)}},\quad r\in[d],c\in[k],$$ so that by definition of tensor decomposition, we get
    \begin{flalign*}
        F_{n_1,n_2,n_3}&=\sum_{j=1}^k[V_d]_{n_1,j}[V_d]_{n_2,j}[V_dD_w]_{n_3,j}\\
        &=\sum_{j=1}^k e^{i\pi\mu_{n_1}^{(j)}}e^{i\pi\mu_{n_2}^{(j)}}w_je^{i\pi\mu_{n_3}^{(j)}}\\
        &=\sum_{j=1}^k w_je^{i\pi(\mu_{n_1}^{(j)}+\mu_{n_2}^{(j)}+\mu_{n_3}^{(j)})}=f(e_1+e_2+e_3),
    \end{flalign*} as required.
\end{proof}
\chapter{Main algorithm}
Let us recall what the goal of the problem is. Broadly speaking, super-resolution aims to recover a superposition of point sources using bandlimited measurements that may be corrupted with noise. In this setting, the created algorithm works in the Fourier domain where each of the $k$ points of the $d$-dimensional plane are separated by a distance at least $\Delta$. Hence, the frequencies of the Fourier measurements are bounded by $O(1/\Delta)$. The idea of the procedure is to take random bandlimited measurements with cutoff frequency bounded by $\Omega(\sqrt{d}/\Delta)$ and perform Tensor decomposition to recover the estimates of the point sources.\paragraph{Input and output} The algorithm takes as input a cutoff frequency $R$, the defined number of measurements $m$ and a noisy measurement function $\tilde{f}(\cdot)$ and outputs the set of estimates $$\{\widehat{w}_j,\widehat{\mu}^{(j)}:j\in[k]\},$$ where the $\widehat{w}_j$'s are the complex weight coefficients and the $\widehat{\mu}^{(j)}$'s are the estimates of the point sources. Note that in case the noise is non-existant (i.e. when $\epsilon_z=0$), the parameters are recovered exactly. Otherwise, stable recovery implies that the estimates are such that $$\min_\pi\max\left\{||\widehat{\mu}^{(j)}-\mu^{(\pi(j))}||_2:j\in[k]\right\}\leq poly(d,k)\epsilon_z.$$ In words, the estimates of the point sources differ from their real values by at most the noise $\epsilon_z$ scaled by a polynomial function that depends on the number of dimensions $d$ and the number of point sources $k$.
\paragraph{Measurements}
\chapter{Discussion}
\section{Improvements already made}
The algorithm proposed in the previous chapter allows us to efficiently recover the position of some image source points using only a polynomial amount of space. Recall that the problem instance has multiple base parameters:
\begin{enumerate}
    \item the number of dimensions $d$ needed to represent the entirety of the fine details of the image,
    \item the number of point sources $k$ we wish to recover,
    \item the minimal distance $\Delta$ between any two pair of points on the plane in the $L_2$ sense,
    \item the number $m$ of coarse measurements we have to make in order to efficiently recover the point sources in the noisy case.
\end{enumerate}
The cutoff frequency $R$ of the measurements is in this setting inversely proportional to the minimum distance $\Delta$, that is, $R=O(1/\Delta)$. Compared to previous work which assumed that $R$ could be as large as $\Omega(\sqrt{d}/\Delta)$, data can now be super-resolved well enough with a pretty small cutoff frequency and hence induces a robust enough procedure. Moreover, the number of measurements needed for the procedure to work in a stable enough way do not depend on the distance between the points anymore, but only depend on the number of points and the overall dimension. Indeed previous results required an exponential number of measurements to be taken on the hypergrid, whereas the current procedure's computational complexity is bounded by a polynomial in both $d$ and $k$. This hence results in an exponential improvement.
\section{Stability and efficiency}
To complement what has been stated in previous sections, we reiterate on the fact that stability guarantees of the procedure depend on one element: the condition number $\cond(V)$ of the random Vandermonde matrix. Since the measurements are stored in a tensor, we seek the stability of the tensor decomposition procedure. For it to be stable, we need $\cond(V)$ to be close to 1 in expectation; this happens if the measurements follow a Gaussian distribution.
\paragraph{Efficiency of Vandermonde matrices} Many studies have already shown that Vandermonde matrices tend to be badly ill-conditioned in most scenarios, with the exception of some specific cases such as DFT matrices or as it turns out, Gaussian valued matrices \cite{vandermondeMatrices}.\par 
Recall that the condition number of a matrix governs its tolerance to noisy data. Specifically, it measures how sensitive a particular matrix is to perturbations that can appear in the input data. Although an ill-conditioned matrix augments the sensitivity of its inverse, the eigenvalue problem can stay well conditioned. In other words, it can happen that a matrix is poorly conditioned for inversion whilst the eigenvalue problem is well conditioned, or vice versa. Hence, focusing on the condition number of the measurement matrix is a key element to not neglect whilst wanting to improve the sample complexity of the procedure.
\section{Improvements left to make}
The open problem related to this algorithm is to reduce the sample complexity from quadratic to linear. Namely, we seek a reduction of the number measurements made, from $O(2(m')^2)$ to $O(m')$. Recall that $m'=m+d+1$ corresponds to the first two dimensions of our measurement tensor, $m$ of which are Gaussian, the others respectively being the $d$ basis vectors needed to recover the location of the point sources and a row of ones used for normalisation purposes. If we wish to take a linear amount of measurements, the procedure to generate them will have to change, to wit, we remove the convolution of the two Gaussians. 
\chapter{Conclusion}
The goal of this project was to familiarise myself with a topic called super-resolution by reading a recent paper on the subject. The paper in question called "Super resolution off the grid" that was written in 2015 by Qingqing Huang and Sham Kakade highlighted an algorithm that is capable to efficiently recover the location of some spikes of a low resolution image.\par

\printbibliography

\appendix
\chapter{Construction of Hankel matrices}
\label{appendix:hankel}
In this section, we prove that the Hankel matrices $H_0$ and $H_1$ defined earlier admit the respective diagonalisations $V_mD_wV_m^T$ and $V_mD_wD_\mu V_m^T$. To do so, we first compute the product $D_wV_m^T$. We have 
\begin{flalign*}
    D_wV_m^T&=
    \begin{pmatrix}
        w_1&0&\ldots&0\\
        0&w_2&\ldots&0\\
        \vdots&\vdots&\ddots&\vdots\\
        0&0&\ldots&w_k
    \end{pmatrix}
    \begin{pmatrix}
        1&e^{i\pi\mu^{(1)}}&\ldots&\big(e^{i\pi\mu^{(1)}}\big)^{m-1}\\
        1&e^{i\pi\mu^{(2)}}&\ldots&\big(e^{i\pi\mu^{(2)}}\big)^{m-1}\\
        \vdots&\vdots&\ddots&\vdots\\
        1&e^{i\pi\mu^{(k)}}&\ldots&\big(e^{i\pi\mu^{(k)}}\big)^{m-1}
    \end{pmatrix}\\
    &=\begin{pmatrix}
        w_1&w_1e^{i\pi\mu^{(1)}}&\ldots&w_1\big(e^{i\pi\mu^{(1)}}\big)^{m-1}\\
        w_2&w_2e^{i\pi\mu^{(2)}}&\ldots&w_2\big(e^{i\pi\mu^{(2)}}\big)^{m-1}\\
        \vdots&\vdots&\ddots&\vdots\\
        w_k&w_ke^{i\pi\mu^{(k)}}&\ldots&w_k\big(e^{i\pi\mu^{(k)}}\big)^{m-1}
    \end{pmatrix},
\end{flalign*} so that
\begin{flalign*}
    H_0=V_mD_wV_m^T&=
    \begin{pmatrix}
        1&\ldots&1\\
        e^{i\pi\mu^{(1)}}&\ldots&e^{i\pi\mu^{(k)}}\\
        \vdots&\cdots&\vdots\\
        \big(e^{i\pi\mu^{(1)}}\big)^{m-1}&\ldots&\big(e^{i\pi\mu^{(k)}}\big)^{m-1}
    \end{pmatrix}
    \begin{pmatrix}
        w_1&w_1e^{i\pi\mu^{(1)}}&\ldots&w_1\big(e^{i\pi\mu^{(1)}}\big)^{m-1}\\
        w_2&w_2e^{i\pi\mu^{(2)}}&\ldots&w_2\big(e^{i\pi\mu^{(2)}}\big)^{m-1}\\
        \vdots&\vdots&\ddots&\vdots\\
        w_k&w_ke^{i\pi\mu^{(k)}}&\ldots&w_k\big(e^{i\pi\mu^{(k)}}\big)^{m-1}
    \end{pmatrix}\\
    &=\begin{pmatrix}
        \sum_{j\in[k]}w_j&\sum_{j\in[k]}w_je^{i\pi\mu^{(1)}}&\ldots&\sum_{j\in[k]}w_j\big(e^{i\pi\mu^{(1)}}\big)^{m-1}\\
        \sum_{j\in[k]}w_je^{i\pi\mu^{(1)}}&\sum_{j\in[k]}w_j\big(e^{i\pi\mu^{(1)}}\big)^2&\ldots&\sum_{j\in[k]}w_j\big(e^{i\pi\mu^{(1)}}\big)^m\\
        \vdots&\vdots&\ddots&\vdots\\
        \sum_{j\in[k]}w_j\big(e^{i\pi\mu^{(1)}}\big)^{m-1}&\sum_{j\in[k]}w_j\big(e^{i\pi\mu^{(1)}}\big)^m&\ldots&\sum_{j\in[k]}w_j\big(e^{i\pi\mu^{(1)}}\big)^{2m-1}\\
    \end{pmatrix}\\
    &=\begin{pmatrix}
        f(0)&f(1)&\ldots&f(m-1)\\
        f(1)&f(2)&\ldots&f(m)\\
        \vdots&\vdots&\ddots&\vdots\\
        f(m-1)&f(m)&\ldots&f(2m-1)
    \end{pmatrix},
\end{flalign*} 
where $f(s)=\sum_{j\in[k]}w_j\big(e^{i\pi\mu^{(j)}}\big)^s$ for $s=0,\ldots,2m-1$ which indeed corresponds to the defined Hankel matrix. The proof is similar for $H_1$ except that $s$ varies from 1 to $2m$ since element-wise multiplication of $H_0$ with the diagonal matrix $D_\mu$ yields a simple scaling by a factor $e^{i\pi\mu^{(j)}}$.

\clearpage
\pagestyle{numberonly}


\end{document}