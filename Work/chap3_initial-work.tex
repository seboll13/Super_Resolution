\chapter{Initial work}
\section{Prony's method}
In 1795, the french mathematician Gaspard de Prony came up with a very useful trick to solve a recovery problem, which aims to reconstruct functions from their values at given points. Essentially, we can view those functions as an $n$'th degree polynomial from which we want to recover the roots. This is today called \textit{Prony's method} and can easily be applied to the theory of super-resolution \cite{pronyMethod}.
\paragraph{General form} In a general sense, consider functions of the form $z=f(x)=\sum_{j\in[n]}\mu_j\rho_j^x$, where $\mu_j\in\mathbb{R}$ and $\rho_j\in\mathbb{R}_+$ for every $1\leq i,j\leq n$. We want to recover those parameters from the samples $z_k:=f(x_k)$ of $f$. We are faced with what is called Prony's condition, which states that given an undetermined number of measurements or sampling points $z_0,z_1,\ldots$, we impose that $$(z*p)_k=\sum_{j=0}^n z_{k+1}p_j=0,$$ for $k=0,1,2,\ldots$ and we wish to solve the above equations for the $p_j$'s. We see that the solutions are given by $p(x)=(x-\widehat{\rho}_1)\cdots(x-\widehat{\rho}_n)$. In a more algorithmic sense, we seek to 
\begin{enumerate}
    \item solve the above linear system for $p$ and 
    \item find the zeroes of $p(x)$ to obtain the $\rho_j$'s.
\end{enumerate}
\paragraph{Link with super-resolution} Prony's method can in fact be applied to many domains. For example in microscopy, images often contain brightness spikes on the plane but the optical system that contains all the hardware to create the picture essentially acts as a low pass filter: it will keep the low frequencies of a signal and attenuate the frequencies that are higher than a pre-determined cutoff frequency. By doing so, those spikes are turned into what are called \textit{point spread functions}, that determine the performance of an imaging system. Similarly to our problem, the goal of Prony's method is to reconstruct the image by recovering estimates of the original spikes.\par 
We can view the plane of an image as part of $\mathbb{R}^2$. Let its spikes be at positions $X\subset\mathbb{R}^2$ and let them have amplitude $a_x$. Note that $X$ is sparse. Taking the Fourier transform yields $$z_\alpha:=z_{\alpha_1,\alpha_2}=\sum_{x\in X}a_xe^{\alpha_1 x_1+\alpha_2 x_2},$$ where $\alpha=(\alpha_1,\alpha_2)\in\mathbb{Z}^2$. Through the low pass filter, we obtain a finite amount of measurements $$z_\alpha=\sum_{x\in X}a_x\rho_x^\alpha,$$ for $|\alpha_1|,|\alpha_2|\leq n$ where $\rho_x^\alpha=\rho_{x,1}^{\alpha_1}\rho_{x,2}^{\alpha_2}$ and $\rho_{x,j}=e^{ix_j}$ for $j=1,2$. Note that the recovery of the $z_\alpha$'s is exactly the 2D version of Prony's problem.
\section{Univariate case: adaptation of Prony's method}
The initial setting is as follows. We construct two $m\times m$ complex valued Hankel matrices $H_0$ and $H_1$, that is, matrices such that their skew-diagonals\footnote{A skew-diagonal is the diagonal in the North-East direction.} are constants. We have $D_w\in\mathbb{C}^{k\times k}$ where $[D_w]_{j,j}=w_j$ and $D_\mu\in\mathbb{C}^{k\times k}$ where $[D_\mu]_{j,j}=e^{i\pi\mu^{(j)}}$. We furthermore have a Vandermonde matrix $V_m\in\mathbb{C}^{m\times k}$ defined as $$V_m=
\begin{pmatrix}
    1&\ldots&1\\
    \big(e^{i\pi\mu^{(1)}}\big)^1&\ldots&\big(e^{i\pi\mu^{(k)}}\big)^1\\
    \vdots&\ldots&\vdots\\
    \big(e^{i\pi\mu^{(1)}}\big)^{m-1}&\ldots&\big(e^{i\pi\mu^{(k)}}\big)^{m-1}
\end{pmatrix}$$ We construct a 3rd order tensor $F\in\mathbb{C}^{m\times m\times 2}$ in which $F_{i,i',j}=[H_{j-1}]_{i,i'}$ for $j=1,2$ and $i,i'\in[m]$.
\begin{fact}
    In the setting above, $F$ admits the unique rank $k$ tensor decomposition $F=V_m\otimes V_m\otimes (V_2D_w)$.
\end{fact}
\begin{proof}
    We start by computing the product $V_2D_w$. By definition of $V_m$, we have that $[V_2]_{r,c}=\big(e^{i\pi\mu^{(c)}}\big)^{r-1}$ for $r\in[2]$ and $c\in[k]$. Multiplying with the diagonal matrix $D_w$ yields $$[V_2D_w]_{r,c}=w_c\big(e^{i\pi\mu^{(c)}}\big)^{r-1},\quad\forall r\in[2],c\in[k].$$
    We aim to show that 
    \begin{flalign*}
        F_{i_1,i_2,i_3}&=\sum_{n=1}^k[V_m]_{i_1,n}[V_m]_{i_2,n}[V_2D_w]_{i_3,n}\\
        &=\sum_{n=1}^k\big(e^{i\pi\mu^{(n)}}\big)^{i_1-1}\big(e^{i\pi\mu^{(n)}}\big)^{i_2-1}w_n\big(e^{i\pi\mu^{(n)}}\big)^{i_3-1}\\
        &=\sum_{n=1}^k w_n\big(e^{i\pi\mu^{(n)}}\big)^{i_1+i_2+i_3-3},
    \end{flalign*} for $i_1,i_2\in[m]$ and $i_3\in[2]$. It can be verified that the measurements that form both Hankel matrices are given by $$f(s)=\sum_{j\in[k]}w_j\big(e^{i\pi\mu^{(j)}}\big)^s,$$ with $s=0,\ldots,2m-1$ for $H_0$ and $s=1,\ldots,2m$ for $H_1$. For the details of the proof, see Appendix \ref{appendix:hankel}. Then, since $F_{i_1,i_2,i_3}=[H_{i_3-1}]_{i_1,i_2}$, for a fixed $i_3$ we have either $F_{i_1,i_2,1}=[H_0]_{i_1,i_2}$ or $F_{i_1,i_2,2}=[H_1]_{i_1,i_2}$. Note that since we are dealing with Hankel matrices, $[H_0]_{i_1,i_2}=f(i_1+i_2-2)$. Hence,
    $$F_{i_1,i_2,i_3}=\sum_{n\in[k]}w_n\big(e^{i\pi\mu^{(n)}}\big)^{i_1+i_2+i_3-3}=f(i_1+i_2+i_3-3),$$ as required.
\end{proof}
\section{Multivariate toy case}
We follow the definitions given in the paper. We have $D_w=Diag(w_j)\in\mathbb{C}^{k\times k}$ and $[V_d]_{r,c}=e^{i\pi\mu_r^{(c)}}\in\mathbb{C}^{d\times k}$ the latter in which $\mu_n^{(i)}$ are i.i.d. $\sim Unif([-1,+1])$. Furthermore, we have $F_{n_1,n_2,n_3}=f(s)\big|_{s=e_{n_1}+e_{n_2}+e_{n_3}},$ for all $n_1,n_2,n_3\in[d]$. We have the following fact:
\begin{fact}
    In the setting above, $F$ admits the tensor decomposition $F=V_d\otimes V_d\otimes(V_dD_w)$.
\end{fact}
\begin{proof}
    We wish to show that $f(s)\big|_{s=s^{(n_1)}+s^{(n_2)}+s^{(n_1)}}=\sum_{j=1}^k w_je^{i\pi(\mu_{n_1}^{(j)}+\mu_{n_2}^{(j)}+\mu_{n_3}^{(j)})}$. To do so, we first compute the matrix product $V_dD_w$. Since $[V_d]_{r,c}=e^{i\pi\mu_r^{(c)}}$ for $r\in[d],c\in[k]$ and $D_w=Diag(w_j)$, we directly have that $$[V_dD_w]_{r,c}=w_ce^{i\pi\mu_r^{(c)}},\quad r\in[d],c\in[k],$$ so that by definition of tensor decomposition, we get
    \begin{flalign*}
        F_{n_1,n_2,n_3}&=\sum_{j=1}^k[V_d]_{n_1,j}[V_d]_{n_2,j}[V_dD_w]_{n_3,j}\\
        &=\sum_{j=1}^k e^{i\pi\mu_{n_1}^{(j)}}e^{i\pi\mu_{n_2}^{(j)}}w_je^{i\pi\mu_{n_3}^{(j)}}\\
        &=\sum_{j=1}^k w_je^{i\pi(\mu_{n_1}^{(j)}+\mu_{n_2}^{(j)}+\mu_{n_3}^{(j)})}\\
        &=f(s)\big|_{s=s^{(n_1)}+s^{(n_2)}+s^{(n_1)}}=f(e_{n_1}+e_{n_2}+e_{n_3}),
    \end{flalign*} as required.
\end{proof}