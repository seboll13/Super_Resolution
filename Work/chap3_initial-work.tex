\chapter{Initial work}
\section{Prony's method}
In this section, we will consider 
\section{Univariate case: adaptation of Prony's method}
\section{Multivariate toy case}
We follow the definitions given in the paper. We have $D_w=Diag(w_j)\in\mathbb{C}^{k\times k}$ and $[V_d]_{r,c}=e^{i\pi\mu_r^{(c)}}\in\mathbb{C}^{d\times k}$ the latter in which $\mu_n^{(i)}$ are i.i.d. $\sim Unif([-1,+1])$. Furthermore, we have $F_{n_1,n_2,n_3}=f(s)\big|_{s=e_{n_1}+e_{n_2}+e_{n_3}},$ for all $n_1,n_2,n_3\in[d]$. We have the following fact:
\begin{fact}
    In the setting above, $F$ admits the tensor decomposition $F=V_d\otimes V_d\otimes(V_dD_w)$.
\end{fact}
\begin{proof}
    We wish to show that $f(e_1+e_2+e_3)=\sum_{j=1}^k w_je^{i\pi(\mu_1^{(j)}+\mu_2^{(j)}+\mu_3^{(j)})}$. To do so, we first compute the matrix product $V_dD_w$. We have
    \begin{flalign*}
        V_dD_w&=\begin{pmatrix}
            e^{i\pi\mu_1^{(1)}}&\ldots&e^{i\pi\mu_1^{(k)}}\\
            \vdots&\ddots&\vdots\\
            e^{i\pi\mu_d^{(1)}}&\ldots&e^{i\pi\mu_d^{(k)}}
        \end{pmatrix}
        \begin{pmatrix}
            w_1&0&\ldots&0\\
            0&w_2&\ldots&0\\
            \vdots&\vdots&\ddots&\vdots\\
            0&0&\ldots&w_k
        \end{pmatrix}\\&=
        \begin{pmatrix}
            w_1e^{i\pi\mu_1^{(1)}}&w_2e^{i\pi\mu_1^{(2)}}&\ldots&w_ke^{i\pi\mu_1^{(k)}}\\
            w_1e^{i\pi\mu_2^{(1)}}&w_2e^{i\pi\mu_2^{(2)}}&\ldots&w_ke^{i\pi\mu_2^{(k)}}\\
            \vdots&\vdots&\ddots&\vdots\\
            w_1e^{i\pi\mu_d^{(1)}}&w_2e^{i\pi\mu_d^{(2)}}&\ldots&w_ke^{i\pi\mu_d^{(k)}}
        \end{pmatrix},
    \end{flalign*} so that by definition of tensor decomposition, we get
    \begin{flalign*}
        F_{n_1,n_2,n_3}&=\sum_{j=1}^k[V_d]_{n_1,j}[V_d]_{n_2,j}[V_dD_w]_{n_3,j}\\
        &=\sum_{j=1}^k e^{i\pi\mu_{n_1}^{(j)}}e^{i\pi\mu_{n_2}^{(j)}}w_je^{i\pi\mu_{n_3}^{(j)}}\\
        &=\sum_{j=1}^k w_je^{i\pi(\mu_{n_1}^{(j)}+\mu_{n_2}^{(j)}+\mu_{n_3}^{(j)})}=f(e_1+e_2+e_3),
    \end{flalign*} as required.
\end{proof}